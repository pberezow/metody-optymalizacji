\documentclass{article}
\title{Metody optymalizacji. Lista 3}
\author{Piotr Berezowski, 236749}

\usepackage{polski}
\usepackage[utf8]{inputenc}
\usepackage{enumerate}
\usepackage{graphicx}
\usepackage{subfig}
\usepackage{cases}
\usepackage{mathtools}
\usepackage{float}
\usepackage{cleveref}
\graphicspath{ {./src/} }

\begin{document}
	\maketitle
	\pagenumbering{gobble}
	\newpage
	\pagenumbering{arabic}
	
	\section{Treść zadania}

    Zaimplementować w języku \textit{julia} z użyciem pakietu \textit{JuMP} algorytm aproksymacyjny oparty na programowaniu liniowym dla 
    uogólnionego zagadnienia przydziału (the generalized assignment problem). 
    
    Ocenić eksperymentalnie jakość proponowanego algorytmu aproksymacyjnego dla znacznej części danych z OR-Library.
    
    \section{Problem}

    Niech $M$ oznacza zbiór maszyn, a $J$ oznacza zbiór zadań. Każda maszyna może zostać przypisana do każdego zadania produkując w ten sposób 
    koszt zależny od maszyny i zadania. Dodatkowo każda maszyna ma określoną maksymalną ilość zasobów jakie może rozdysponować pomiędzy zadania, 
    i nie może jej przekroczyć. Ilość zasobów jakie zajmuje konkretne zadanie różni się w zależności od maszyny do której jest ono przypisane. 

    Uogólnione zagadnienie przydziału polega na znalezieniu takiego przypisania wszystkich zadań należących do $J$ do maszyn należących 
    do $M$, aby koszt całkowity był minimalny, przy jednoczesnym nieprzekroczeniu maksymalnej pojemności żadnej z maszyn. 
    
    \section{Model}

    Uogólnione zagadnienie przydziału możemy zdefiniować jako zadanie programowania całkowitoliczbowego w następujący sposób:

    Niech $x_{ij}$, $i \in M,\ j \in J$ będzie zmienną decyzyjną przyjmującą wartość ze zbioru $\{0, 1\}$. Jeśli zadanie $j$ jest 
    przypisane do maszyny $i$, to $x_{ij} = 1$, w przeciwnym przypadku $x_{ij} = 0$.

    Niech $c_{ij}$ oznacza koszt przypisania $j$-tego zadania do $i$-tej maszyny.

    Niech $p_{ij}$ określa ilość zasobów jaką zużywa rozwiązanie zadania $j$ na maszynie $i$.

    Niech $T_i$ oznacza maksymalną ilość zasobów dla $i$-tej maszyny.

    Funkcja celu ma następującą postać:

    $$\textnormal{minimize } \sum_{i \in M} \sum_{j \in J} x_{ij} c_{ij}$$

    Ograniczenia, które muszą zostać spełnione:
    \begin{itemize}
        \item Nie można przekroczyć maksymalnej pojemności maszyn:
            $$\sum_{j \in J} x_{ij} p_{ij} \leq T_i, \ \textnormal{dla wszystkich } i \in M$$
        \item Przypisanie każdego z zadań do jednej maszyny:
            $$\sum_{i \in M} x_{ij} = 1, \ \textnormal{dla wszystkich } j \in J$$
        \item Ograniczenia na zmienne decyzyjne:
            $$x_{ij} \in \{0, 1\}, \ \textnormal{dla wszystkich } i \in M, \ j \in J$$
    \end{itemize}

    \section{Algorytm aproksymacyjny}

    W algorytmie zastosowano relaksację modelu całkowitoliczbowego. Ograniczenia na zmienne decyzyjne zastępiono ograniczeniami 
    
    $$0 \leq x_{ij} \leq 1, \ \textnormal{dla wszystkich } i \in M, \ j \in J$$

    Dzięki usunięciu ograniczeń całkowitoliczbowych problem NP-trudny zmienił się w rozwiązywalny w czasie wielomianowym (programowanie liniowe). 

    Ostateczny algorytm jest algorytmem iteracyjnym. W każdej iteracji rozwiązujemy relaksację problemu i budujemy na jego podstawie 
    rozwiązanie częściowe. Następnie uaktualniamy problem, usuwając z niego przydzielone zadania, zmniejszamy odpowiednio pojemność maszyn 
    i usuwamy część maszyn.

    Algorytm wygląda w następujący sposób:

    \begin{enumerate}
        \item Inicjalizacja $F \gets \emptyset, \ M' \gets M$
        \item Dopóki $J \neq \emptyset$:
            \begin{enumerate}
                \item Rozwiąż relaksację modelu całkowitoliczbowego zadania. Usuń wszystkie zmienne decyzyjne $x_{ij} = 0$.
                \item Dla zmiennych decyzyjnych $x_{ij} = 1$ uaktualnij $F \gets F \bigcup \{i, j\}$, $J \gets J \setminus \{j\}$, $T_i \gets T_i - p_{ij}$.
                \item Jeśli istnieje maszyna $i$, której stopień jest równy 1, lub której stopień jest równy 2 i spełnia $\sum_{j \in J} x_{ij} \geq 1$, 
                    wtedy uaktualnij $M' \gets M' \setminus \{i\}$.
            \end{enumerate}
        \item Zwróć rozwiązanie $F$.
    \end{enumerate}

    Rozwiązanie wyznaczone dla opisanego algorytmu jest zawsze nie gorsze niż rozwiązanie optymalne dla oryginalnego problemu, ale 
    istnieje możliwość przekroczenia pojemności maszyn. Wiemy jednak, że pojemność może zostać przekroczona maksymalnie dwukrotnie dla każdej 
    z maszyn.

    W kolejnej sekcji zaprezentowano wyniki dla prezentowanego algorytmu aproksymacyjnego, oraz jego zestawienie z wynikami dokładnymi.
    Implementacja algorytmów znajduje się w dołączonym pliku \textit{zad.jl}.

    \section{Wyniki eksperymentów}

    W tabelach poniżej przedstawiono wyniki dla różnych zbiorów danych pochodzących z \textbf{OR-Library}. Każda tabela odpowiada pojedynczemu 
    zbiorowi zadań (pojedynczy plik z danymi). W kolumnie oetykietowanej jako \textit{Wynik dokładny} zaprezentowane zostały wyniki 
    uzyskane przy pomocy solvera \textbf{GLPK} dla zadania sformułowanego jako zagadnienie programowania całkowitoliczbowego. Dokładniejsze sformułowanie 
    modelu znajduje się w sekcji \textit{Model}. 

    Kolumna oznaczona jako \textit{Aproksymacja} przedstawia wyniki uzyskane przy pomocy algorytmu aproksymacyjnego opisanego w sekcji 
    \textit{Algorytm aproksymacyjny}.

    Kolumna $|M|$ zawiera informacje o ilości maszyn w rozważanym egzemplarzu problemu, podobnie kolumna $|J|$ opisuje ilość zadań do przypisania. 
    Dla rozwiązania dokładnego oraz aproksymacji zamieszczono informację o czasie potrzebnym do wyznaczenia rozwiązania (podany w sekundach), 
    wartości funkcji celu dla znalezionego rozwiązania (\textit{koszt}) oraz o stosunku wykorzystanej pojemności maszyny do pojemności całkowitej.
    W przypadku pojemności przedstawiono 3 wyniki - minimalna pojemność (\textit{min}), średnia pojemność liczona ze wszystkich maszyn (\textit{avg}) 
    oraz maksymalna pojemność (\textit{max}). 

    W przypadku wyników dokładnych, dla niektórych egzemplarzy zadań nie udało się otrzymać wyników w zadowalającym czasie. W takim wypadku 
    wyniki dokładne zostały pominięte.

    \begin{table}[H]
        \begin{center}
            \resizebox{\textwidth}{!}{%
            \begin{tabular}{c|c||c|c|c|c|c||c|c|c|c|c|c}
                      &       & \multicolumn{5}{c||}{Wynik dokładny} & \multicolumn{6}{c}{Aproksymacja} \\ \cline{3-7} \cline{8-13}
                $|M|$ & $|J|$ & koszt & czas (s) & \multicolumn{3}{c||}{$cap / cap_{max}$ (*)} & koszt & czas (s) & \multicolumn{3}{c|}{$cap / cap_{max}$ (*)} & \# iteracji \\ \cline{5-7} \cline{10-12} 
                  &  &       &          & min & avg & max                         &       &          & min & avg & max                                        & \\
                \hline
                \hline

                5 & 15 & 261.0 & 0.0093 & 0.815 & 0.912 & 0.947 & 249.0 & 0.009 & 0.909 & 1.114 & 1.368 & 4 \\ 
                5 & 15 & 269.0 & 0.1737 & 0.789 & 0.905 & 0.955 & 249.0 & 0.0065 & 0.773 & 1.04 & 1.271 & 4 \\ 
                5 & 15 & 256.0 & 0.0149 & 0.914 & 0.958 & 1.0 & 245.0 & 0.0079 & 0.6 & 0.995 & 1.459 & 4 \\ 
                5 & 15 & 274.0 & 0.028 & 0.842 & 0.915 & 0.973 & 260.0 & 0.0058 & 0.842 & 1.077 & 1.278 & 4 \\ 
                5 & 15 & 251.0 & 0.0088 & 0.895 & 0.925 & 0.971 & 243.0 & 0.0062 & 0.743 & 1.159 & 1.647 & 5 \\ 


            \end{tabular}
            }
        \end{center}
        \caption{Wyniki dla przypadków testowych z pliku \textit{gap1.txt}. (*) $cap / cap_{max}$ przedstawia stosunek użytej pojemności maszyny 
            w wyznaczonym rozwiązaniu do maksymalnej pojemności maszyny. W tabeli przedstawiono minimalny, średni i maksymalny stosunek.}
        \label{wyniki-1}
    \end{table}

    \begin{table}[H]
        \begin{center}
            \resizebox{\textwidth}{!}{%
            \begin{tabular}{c|c||c|c|c|c|c||c|c|c|c|c|c}
                      &       & \multicolumn{5}{c||}{Wynik dokładny} & \multicolumn{6}{c}{Aproksymacja} \\ \cline{3-7} \cline{8-13}
                $|M|$ & $|J|$ & koszt & czas (s) & \multicolumn{3}{c||}{$cap / cap_{max}$ (*)} & koszt & czas (s) & \multicolumn{3}{c|}{$cap / cap_{max}$ (*)} & \# iteracji \\ \cline{5-7} \cline{10-12} 
                  &  &       &          & min & avg & max                         &       &          & min & avg & max                                        & \\
                \hline
                \hline

                5 & 20 & 277.0 & 0.0335 & 0.885 & 0.955 & 1.0 & 259.0 & 0.0094 & 0.872 & 1.071 & 1.167 & 5 \\ 
                5 & 20 & 269.0 & 0.052 & 0.907 & 0.973 & 1.0 & 255.0 & 0.0062 & 0.902 & 1.042 & 1.239 & 4 \\ 
                5 & 20 & 260.0 & 0.0388 & 0.922 & 0.949 & 1.0 & 249.0 & 0.0092 & 0.8 & 1.043 & 1.312 & 4 \\ 
                5 & 20 & 269.0 & 0.0395 & 0.904 & 0.945 & 0.981 & 250.0 & 0.008 & 0.961 & 1.148 & 1.269 & 5 \\ 
                5 & 20 & 267.0 & 0.002 & 0.952 & 0.969 & 1.0 & 255.0 & 0.0058 & 0.796 & 1.074 & 1.311 & 3 \\ 

            \end{tabular}
            }
        \end{center}
        \caption{Wyniki dla przypadków testowych z pliku \textit{gap2.txt}. (*) $cap / cap_{max}$ przedstawia stosunek użytej pojemności maszyny 
            w wyznaczonym rozwiązaniu do maksymalnej pojemności maszyny. W tabeli przedstawiono minimalny, średni i maksymalny stosunek.}
        \label{wyniki-2}
    \end{table}

    \begin{table}[H]
        \begin{center}
            \resizebox{\textwidth}{!}{%
            \begin{tabular}{c|c||c|c|c|c|c||c|c|c|c|c|c}
                      &       & \multicolumn{5}{c||}{Wynik dokładny} & \multicolumn{6}{c}{Aproksymacja} \\ \cline{3-7} \cline{8-13}
                $|M|$ & $|J|$ & koszt & czas (s) & \multicolumn{3}{c||}{$cap / cap_{max}$ (*)} & koszt & czas (s) & \multicolumn{3}{c|}{$cap / cap_{max}$ (*)} & \# iteracji \\ \cline{5-7} \cline{10-12} 
                  &  &       &          & min & avg & max                         &       &          & min & avg & max                                        & \\
                \hline
                \hline

                5 & 25 & 438.0 & 0.1387 & 0.948 & 0.977 & 0.984 & 428.0 & 0.0087 & 0.933 & 1.024 & 1.156 & 4 \\ 
                5 & 25 & 415.0 & 0.0315 & 0.868 & 0.954 & 1.0 & 408.0 & 0.0613 & 0.885 & 1.113 & 1.208 & 5 \\ 
                5 & 25 & 446.0 & 0.1385 & 0.954 & 0.987 & 1.0 & 436.0 & 0.0096 & 0.875 & 1.048 & 1.196 & 5 \\ 
                5 & 25 & 430.0 & 0.1705 & 0.899 & 0.973 & 1.0 & 421.0 & 0.0078 & 0.925 & 1.021 & 1.172 & 4 \\ 
                5 & 25 & 411.0 & 0.0275 & 0.847 & 0.931 & 0.981 & 406.0 & 0.0079 & 0.811 & 1.105 & 1.375 & 4 \\ 
                
            \end{tabular}
            }
        \end{center}
        \caption{Wyniki dla przypadków testowych z pliku \textit{gap3.txt}. (*) $cap / cap_{max}$ przedstawia stosunek użytej pojemności maszyny 
            w wyznaczonym rozwiązaniu do maksymalnej pojemności maszyny. W tabeli przedstawiono minimalny, średni i maksymalny stosunek.}
        \label{wyniki-3}
    \end{table}

    \begin{table}[H]
        \begin{center}
            \resizebox{\textwidth}{!}{%
            \begin{tabular}{c|c||c|c|c|c|c||c|c|c|c|c|c}
                      &       & \multicolumn{5}{c||}{Wynik dokładny} & \multicolumn{6}{c}{Aproksymacja} \\ \cline{3-7} \cline{8-13}
                $|M|$ & $|J|$ & koszt & czas (s) & \multicolumn{3}{c||}{$cap / cap_{max}$ (*)} & koszt & czas (s) & \multicolumn{3}{c|}{$cap / cap_{max}$ (*)} & \# iteracji \\ \cline{5-7} \cline{10-12} 
                  &  &       &          & min & avg & max                         &       &          & min & avg & max                                        & \\
                \hline
                \hline

                5 & 30 & 423.0 & 0.4939 & 0.944 & 0.986 & 1.0 & 411.0 & 0.0096 & 0.759 & 1.023 & 1.222 & 4 \\ 
                5 & 30 & 424.0 & 0.9735 & 0.947 & 0.981 & 1.0 & 407.0 & 0.0101 & 0.681 & 1.044 & 1.21 & 4 \\ 
                5 & 30 & 426.0 & 0.3826 & 0.972 & 0.989 & 1.0 & 407.0 & 0.0059 & 0.886 & 1.055 & 1.218 & 3 \\ 
                5 & 30 & 395.0 & 0.4293 & 0.955 & 0.977 & 1.0 & 380.0 & 0.0081 & 0.93 & 1.046 & 1.128 & 4 \\ 
                5 & 30 & 406.0 & 0.3569 & 0.914 & 0.957 & 0.985 & 382.0 & 0.0106 & 1.029 & 1.113 & 1.2 & 5 \\ 
                
            \end{tabular}
            }
        \end{center}
        \caption{Wyniki dla przypadków testowych z pliku \textit{gap4.txt}. (*) $cap / cap_{max}$ przedstawia stosunek użytej pojemności maszyny 
            w wyznaczonym rozwiązaniu do maksymalnej pojemności maszyny. W tabeli przedstawiono minimalny, średni i maksymalny stosunek.}
        \label{wyniki-4}
    \end{table}

    \begin{table}[H]
        \begin{center}
            \resizebox{\textwidth}{!}{%
            \begin{tabular}{c|c||c|c|c|c|c||c|c|c|c|c|c}
                      &       & \multicolumn{5}{c||}{Wynik dokładny} & \multicolumn{6}{c}{Aproksymacja} \\ \cline{3-7} \cline{8-13}
                $|M|$ & $|J|$ & koszt & czas (s) & \multicolumn{3}{c||}{$cap / cap_{max}$ (*)} & koszt & czas (s) & \multicolumn{3}{c|}{$cap / cap_{max}$ (*)} & \# iteracji \\ \cline{5-7} \cline{10-12} 
                  &  &       &          & min & avg & max                         &       &          & min & avg & max                                        & \\
                \hline
                \hline

                8 & 24 & 403.0 & 0.1203 & 0.871 & 0.94 & 1.0 & 390.0 & 0.0156 & 0.395 & 1.135 & 1.486 & 5 \\ 
                8 & 24 & 389.0 & 0.0507 & 0.469 & 0.856 & 0.971 & 381.0 & 0.0104 & 0.469 & 1.003 & 1.325 & 4 \\ 
                8 & 24 & 383.0 & 0.1688 & 0.595 & 0.891 & 1.0 & 378.0 & 0.0131 & 0.595 & 0.947 & 1.211 & 4 \\ 
                8 & 24 & 384.0 & 0.0696 & 0.719 & 0.89 & 0.974 & 375.0 & 0.0124 & 0.5 & 1.031 & 1.469 & 3 \\ 
                8 & 24 & 396.0 & 0.5945 & 0.9 & 0.964 & 1.0 & 386.0 & 0.013 & 0.892 & 1.032 & 1.394 & 3 \\ 
                
            \end{tabular}
            }
        \end{center}
        \caption{Wyniki dla przypadków testowych z pliku \textit{gap5.txt}. (*) $cap / cap_{max}$ przedstawia stosunek użytej pojemności maszyny 
            w wyznaczonym rozwiązaniu do maksymalnej pojemności maszyny. W tabeli przedstawiono minimalny, średni i maksymalny stosunek.}
        \label{wyniki-5}
    \end{table}

    \begin{table}[H]
        \begin{center}
            \resizebox{\textwidth}{!}{%
            \begin{tabular}{c|c||c|c|c|c|c||c|c|c|c|c|c}
                      &       & \multicolumn{5}{c||}{Wynik dokładny} & \multicolumn{6}{c}{Aproksymacja} \\ \cline{3-7} \cline{8-13}
                $|M|$ & $|J|$ & koszt & czas (s) & \multicolumn{3}{c||}{$cap / cap_{max}$ (*)} & koszt & czas (s) & \multicolumn{3}{c|}{$cap / cap_{max}$ (*)} & \# iteracji \\ \cline{5-7} \cline{10-12} 
                  &  &       &          & min & avg & max                         &       &          & min & avg & max                                        & \\
                \hline
                \hline

                8 & 32 & 525.0 & 1.3843 & 0.755 & 0.918 & 1.0 & 516.0 & 0.018 & 0.786 & 1.034 & 1.356 & 5 \\ 
                8 & 32 & 527.0 & 6.1483 & 0.843 & 0.956 & 1.0 & 513.0 & 0.0247 & 0.8 & 1.053 & 1.245 & 6 \\ 
                8 & 32 & 519.0 & 1.5917 & 0.736 & 0.898 & 1.0 & 510.0 & 0.0118 & 0.733 & 1.059 & 1.444 & 4 \\ 
                8 & 32 & 516.0 & 0.1575 & 0.804 & 0.916 & 0.98 & 511.0 & 0.0135 & 0.804 & 1.087 & 1.512 & 4 \\ 
                8 & 32 & 521.0 & 0.2499 & 0.906 & 0.966 & 1.0 & 513.0 & 0.0178 & 0.571 & 1.081 & 1.46 & 5 \\ 
                
            \end{tabular}
            }
        \end{center}
        \caption{Wyniki dla przypadków testowych z pliku \textit{gap6.txt}. (*) $cap / cap_{max}$ przedstawia stosunek użytej pojemności maszyny 
            w wyznaczonym rozwiązaniu do maksymalnej pojemności maszyny. W tabeli przedstawiono minimalny, średni i maksymalny stosunek.}
        \label{wyniki-6}
    \end{table}

    \begin{table}[H]
        \begin{center}
            \resizebox{\textwidth}{!}{%
            \begin{tabular}{c|c||c|c|c|c|c||c|c|c|c|c|c}
                      &       & \multicolumn{5}{c||}{Wynik dokładny} & \multicolumn{6}{c}{Aproksymacja} \\ \cline{3-7} \cline{8-13}
                $|M|$ & $|J|$ & koszt & czas (s) & \multicolumn{3}{c||}{$cap / cap_{max}$ (*)} & koszt & czas (s) & \multicolumn{3}{c|}{$cap / cap_{max}$ (*)} & \# iteracji \\ \cline{5-7} \cline{10-12} 
                  &  &       &          & min & avg & max                         &       &          & min & avg & max                                        & \\
                \hline
                \hline

                8 & 40 & 646.0 & 4.5159 & 0.914 & 0.972 & 1.0 & 633.0 & 0.0272 & 0.914 & 1.044 & 1.19 & 6 \\ 
                8 & 40 & 662.0 & 61.2787 & 0.9 & 0.973 & 1.0 & 649.0 & 0.0261 & 0.783 & 1.062 & 1.214 & 4 \\ 
                8 & 40 & 662.0 & 21.3268 & 0.897 & 0.964 & 1.0 & 650.0 & 0.0156 & 0.741 & 1.03 & 1.298 & 4 \\ 
                8 & 40 & 645.0 & 0.3687 & 0.892 & 0.961 & 1.0 & 638.0 & 0.0234 & 0.542 & 1.024 & 1.293 & 5 \\ 
                8 & 40 & 649.0 & 3.6064 & 0.912 & 0.969 & 1.0 & 639.0 & 0.0186 & 0.732 & 1.034 & 1.276 & 4 \\ 
                
            \end{tabular}
            }
        \end{center}
        \caption{Wyniki dla przypadków testowych z pliku \textit{gap7.txt}. (*) $cap / cap_{max}$ przedstawia stosunek użytej pojemności maszyny 
            w wyznaczonym rozwiązaniu do maksymalnej pojemności maszyny. W tabeli przedstawiono minimalny, średni i maksymalny stosunek.}
        \label{wyniki-7}
    \end{table}

    \begin{table}[H]
        \begin{center}
            \resizebox{\textwidth}{!}{%
            \begin{tabular}{c|c||c|c|c|c|c||c|c|c|c|c|c}
                      &       & \multicolumn{5}{c||}{Wynik dokładny} & \multicolumn{6}{c}{Aproksymacja} \\ \cline{3-7} \cline{8-13}
                $|M|$ & $|J|$ & koszt & czas (s) & \multicolumn{3}{c||}{$cap / cap_{max}$ (*)} & koszt & czas (s) & \multicolumn{3}{c|}{$cap / cap_{max}$ (*)} & \# iteracji \\ \cline{5-7} \cline{10-12} 
                  &  &       &          & min & avg & max                         &       &          & min & avg & max                                        & \\
                \hline
                \hline

                8 & 48 & 797.0 & 31.2847 & 0.962 & 0.983 & 1.0 & 786.0 & 0.0225 & 0.68 & 1.016 & 1.192 & 4 \\ 
                8 & 48 & --- & --- & --- & --- & --- & 770.0 & 0.022 & 0.812 & 1.016 & 1.146 & 5 \\ 
                8 & 48 & --- & --- & --- & --- & --- & 786.0 & 0.0242 & 0.894 & 1.028 & 1.204 & 5 \\ 
                8 & 48 & 789.0 & 9.8674 & 0.958 & 0.987 & 1.0 & 782.0 & 0.0195 & 0.804 & 1.011 & 1.17 & 5 \\ 
                8 & 48 & 792.0 & 105.0515 & 0.961 & 0.988 & 1.0 & 778.0 & 0.0232 & 0.88 & 1.041 & 1.211 & 5 \\ 
                
            \end{tabular}
            }
        \end{center}
        \caption{Wyniki dla przypadków testowych z pliku \textit{gap8.txt}. (*) $cap / cap_{max}$ przedstawia stosunek użytej pojemności maszyny 
            w wyznaczonym rozwiązaniu do maksymalnej pojemności maszyny. W tabeli przedstawiono minimalny, średni i maksymalny stosunek.}
        \label{wyniki-8}
    \end{table}

    \begin{table}[H]
        \begin{center}
            \resizebox{\textwidth}{!}{%
            \begin{tabular}{c|c||c|c|c|c|c||c|c|c|c|c|c}
                      &       & \multicolumn{5}{c||}{Wynik dokładny} & \multicolumn{6}{c}{Aproksymacja} \\ \cline{3-7} \cline{8-13}
                $|M|$ & $|J|$ & koszt & czas (s) & \multicolumn{3}{c||}{$cap / cap_{max}$ (*)} & koszt & czas (s) & \multicolumn{3}{c|}{$cap / cap_{max}$ (*)} & \# iteracji \\ \cline{5-7} \cline{10-12} 
                  &  &       &          & min & avg & max                         &       &          & min & avg & max                                        & \\
                \hline
                \hline

                10 & 30 & 482.0 & 1.8879 & 0.8 & 0.932 & 1.0 & 474.0 & 0.0294 & 0.649 & 1.003 & 1.25 & 5 \\ 
                10 & 30 & 476.0 & 0.297 & 0.655 & 0.885 & 1.0 & 468.0 & 0.0253 & 0.806 & 1.054 & 1.237 & 6 \\ 
                10 & 30 & 496.0 & 14.7399 & 0.75 & 0.915 & 1.0 & 481.0 & 0.0151 & 0.639 & 1.096 & 1.613 & 4 \\ 
                10 & 30 & 497.00000000000006 & 64.2967 & 0.711 & 0.932 & 1.0 & 482.0 & 0.0236 & 0.667 & 1.039 & 1.368 & 6 \\ 
                10 & 30 & 488.0 & 7.9434 & 0.816 & 0.958 & 1.0 & 473.0 & 0.0191 & 0.75 & 1.097 & 1.471 & 5 \\ 

            \end{tabular}
            }
        \end{center}
        \caption{Wyniki dla przypadków testowych z pliku \textit{gap9.txt}. (*) $cap / cap_{max}$ przedstawia stosunek użytej pojemności maszyny 
            w wyznaczonym rozwiązaniu do maksymalnej pojemności maszyny. W tabeli przedstawiono minimalny, średni i maksymalny stosunek.}
        \label{wyniki-9}
    \end{table}

    \begin{table}[H]
        \begin{center}
            \resizebox{\textwidth}{!}{%
            \begin{tabular}{c|c||c|c|c|c|c||c|c|c|c|c|c}
                      &       & \multicolumn{5}{c||}{Wynik dokładny} & \multicolumn{6}{c}{Aproksymacja} \\ \cline{3-7} \cline{8-13}
                $|M|$ & $|J|$ & koszt & czas (s) & \multicolumn{3}{c||}{$cap / cap_{max}$ (*)} & koszt & czas (s) & \multicolumn{3}{c|}{$cap / cap_{max}$ (*)} & \# iteracji \\ \cline{5-7} \cline{10-12} 
                  &  &       &          & min & avg & max                         &       &          & min & avg & max                                        & \\
                \hline
                \hline

                10 & 40 & 638.0 & 143.0175 & 0.804 & 0.918 & 1.0 & 626.0 & 0.014 & 0.652 & 1.009 & 1.306 & 3 \\ 
                10 & 40 & 638.0 & 6.2855 & 0.796 & 0.939 & 1.0 & 628.0 & 0.0376 & 0.617 & 1.055 & 1.417 & 6 \\ 
                10 & 40 & 654.0 & 1476.1584 & 0.92 & 0.976 & 1.0 & 640.0 & 0.0183 & 0.583 & 1.045 & 1.364 & 4 \\ 
                10 & 40 & 635.0 & 0.2374 & 0.864 & 0.962 & 1.0 & 627.0 & 0.0205 & 0.727 & 1.019 & 1.408 & 4 \\ 
                10 & 40 & 639.0 & 0.7224 & 0.833 & 0.939 & 1.0 & 631.0 & 0.0253 & 0.756 & 1.003 & 1.157 & 5 \\ 

            \end{tabular}
            }
        \end{center}
        \caption{Wyniki dla przypadków testowych z pliku \textit{gap10.txt}. (*) $cap / cap_{max}$ przedstawia stosunek użytej pojemności maszyny 
            w wyznaczonym rozwiązaniu do maksymalnej pojemności maszyny. W tabeli przedstawiono minimalny, średni i maksymalny stosunek.}
        \label{wyniki-10}
    \end{table}

    \begin{table}[H]
        \begin{center}
            \resizebox{\textwidth}{!}{%
            \begin{tabular}{c|c||c|c|c|c|c||c|c|c|c|c|c}
                      &       & \multicolumn{5}{c||}{Wynik dokładny} & \multicolumn{6}{c}{Aproksymacja} \\ \cline{3-7} \cline{8-13}
                $|M|$ & $|J|$ & koszt & czas (s) & \multicolumn{3}{c||}{$cap / cap_{max}$ (*)} & koszt & czas (s) & \multicolumn{3}{c|}{$cap / cap_{max}$ (*)} & \# iteracji \\ \cline{5-7} \cline{10-12} 
                  &  &       &          & min & avg & max                         &       &          & min & avg & max                                        & \\
                \hline
                \hline

                10 & 50 & 573.0 & 17.3361 & 0.899 & 0.973 & 1.0 & 563.0 & 0.0233 & 0.725 & 1.038 & 1.279 & 4 \\ 
                10 & 50 & 583.0 & 21.1853 & 0.903 & 0.974 & 1.0 & 574.0 & 0.021 & 0.722 & 1.01 & 1.344 & 4 \\ 
                10 & 50 & 589.0 & 5.7646 & 0.94 & 0.981 & 1.0 & 580.0 & 0.0313 & 0.565 & 0.993 & 1.243 & 5 \\ 
                10 & 50 & 578.0 & 632.2081 & 0.877 & 0.95 & 1.0 & 563.0 & 0.0314 & 0.652 & 1.078 & 1.314 & 5 \\ 
                10 & 50 & 581.0 & 1.2922 & 0.783 & 0.927 & 1.0 & 576.0 & 0.0467 & 0.536 & 0.989 & 1.213 & 6 \\ 
                
            \end{tabular}
            }
        \end{center}
        \caption{Wyniki dla przypadków testowych z pliku \textit{gap11.txt}. (*) $cap / cap_{max}$ przedstawia stosunek użytej pojemności maszyny 
            w wyznaczonym rozwiązaniu do maksymalnej pojemności maszyny. W tabeli przedstawiono minimalny, średni i maksymalny stosunek.}
        \label{wyniki-11}
    \end{table}

    \begin{table}[H]
        \begin{center}
            \resizebox{\textwidth}{!}{%
            \begin{tabular}{c|c||c|c|c|c|c||c|c|c|c|c|c}
                      &       & \multicolumn{5}{c||}{Wynik dokładny} & \multicolumn{6}{c}{Aproksymacja} \\ \cline{3-7} \cline{8-13}
                $|M|$ & $|J|$ & koszt & czas (s) & \multicolumn{3}{c||}{$cap / cap_{max}$ (*)} & koszt & czas (s) & \multicolumn{3}{c|}{$cap / cap_{max}$ (*)} & \# iteracji \\ \cline{5-7} \cline{10-12} 
                  &  &       &          & min & avg & max                         &       &          & min & avg & max                                        & \\
                \hline
                \hline

                10 & 60 & --- & --- & --- & --- & --- & 967.0 & 0.0225 & 0.696 & 1.024 & 1.203 & 4 \\ 
                10 & 60 & --- & --- & --- & --- & --- & 944.0 & 0.0374 & 0.689 & 1.009 & 1.181 & 4 \\ 
                10 & 60 & 941.0 & 19.7449 & 0.932 & 0.986 & 1.0 & 932.0 & 0.4628 & 0.905 & 1.063 & 1.242 & 3 \\ 
                10 & 60 & --- & --- & --- & --- & --- & 943.0 & 0.0241 & 0.73 & 1.037 & 1.246 & 4 \\ 
                10 & 60 & 945.0 & 76.0044 & 0.729 & 0.948 & 1.0 & 934.0 & 0.3714 & 0.836 & 1.085 & 1.278 & 5 \\ 
                
            \end{tabular}
            }
        \end{center}
        \caption{Wyniki dla przypadków testowych z pliku \textit{gap12.txt}. (*) $cap / cap_{max}$ przedstawia stosunek użytej pojemności maszyny 
            w wyznaczonym rozwiązaniu do maksymalnej pojemności maszyny. W tabeli przedstawiono minimalny, średni i maksymalny stosunek.}
        \label{wyniki-12}
    \end{table}

    \begin{table}[H]
        \begin{center}
            \resizebox{\textwidth}{!}{%
            \begin{tabular}{c|c||c|c|c|c|c||c|c|c|c|c|c}
                      &       & \multicolumn{5}{c||}{Wynik dokładny} & \multicolumn{6}{c}{Aproksymacja} \\ \cline{3-7} \cline{8-13}
                $|M|$ & $|J|$ & koszt & czas (s) & \multicolumn{3}{c||}{$cap / cap_{max}$ (*)} & koszt & czas (s) & \multicolumn{3}{c|}{$cap / cap_{max}$ (*)} & \# iteracji \\ \cline{5-7} \cline{10-12} 
                  &  &       &          & min & avg & max                         &       &          & min & avg & max                                        & \\
                \hline
                \hline

                5 & 100 & 1698.0 & 0.008 & 0.781 & 0.895 & 0.991 & 1696.0 & 0.0127 & 0.781 & 0.894 & 1.056 & 2 \\ 
                5 & 200 & 3235.0 & 0.0324 & 0.793 & 0.89 & 0.991 & 3234.0 & 0.0251 & 0.784 & 0.889 & 1.025 & 2 \\ 
                10 & 100 & 1360.0 & 0.0547 & 0.38 & 0.777 & 1.0 & 1358.0 & 0.0277 & 0.344 & 0.789 & 1.052 & 3 \\ 
                10 & 200 & 2623.0 & 0.1042 & 0.532 & 0.81 & 1.0 & 2623.0 & 0.0468 & 0.608 & 0.806 & 1.016 & 2 \\ 
                20 & 100 & 1158.0 & 0.0829 & 0.0 & 0.73 & 0.99 & 1157.0 & 0.0446 & 0.05 & 0.736 & 1.09 & 2 \\ 
                20 & 200 & 2339.0 & 0.231 & 0.442 & 0.759 & 1.0 & 2337.0 & 0.1578 & 0.417 & 0.758 & 1.035 & 3 \\ 
                
            \end{tabular}
            }
        \end{center}
        \caption{Wyniki dla przypadków testowych z pliku \textit{gapa.txt}. (*) $cap / cap_{max}$ przedstawia stosunek użytej pojemności maszyny 
            w wyznaczonym rozwiązaniu do maksymalnej pojemności maszyny. W tabeli przedstawiono minimalny, średni i maksymalny stosunek.}
        \label{wyniki-a}
    \end{table}

    \begin{table}[H]
        \begin{center}
            \resizebox{\textwidth}{!}{%
            \begin{tabular}{c|c||c|c|c|c|c||c|c|c|c|c|c}
                      &       & \multicolumn{5}{c||}{Wynik dokładny} & \multicolumn{6}{c}{Aproksymacja} \\ \cline{3-7} \cline{8-13}
                $|M|$ & $|J|$ & koszt & czas (s) & \multicolumn{3}{c||}{$cap / cap_{max}$ (*)} & koszt & czas (s) & \multicolumn{3}{c|}{$cap / cap_{max}$ (*)} & \# iteracji \\ \cline{5-7} \cline{10-12} 
                  &  &       &          & min & avg & max                         &       &          & min & avg & max                                        & \\
                \hline
                \hline

                5 & 100 & 1843.0 & 3.0848 & 0.986 & 0.996 & 1.0 & 1819.0 & 0.0305 & 0.976 & 1.006 & 1.029 & 4 \\ 
                5 & 200 & 3552.0 & 173.3528 & 0.998 & 0.999 & 1.0 & 3537.0 & 0.0525 & 0.972 & 1.004 & 1.032 & 5 \\ 
                10 & 100 & 1407.0 & 16.861 & 0.939 & 0.976 & 1.0 & 1397.0 & 0.0493 & 0.794 & 1.011 & 1.168 & 6 \\ 
                10 & 200 & --- & --- & --- & --- & --- & 2801.0 & 0.1279 & 0.955 & 1.013 & 1.045 & 6 \\ 
                20 & 100 & --- & --- & --- & --- & --- & 1148.0 & 0.0896 & 0.559 & 0.996 & 1.235 & 4 \\ 
                20 & 200 & --- & --- & --- & --- & --- & 2321.0 & 0.2235 & 0.534 & 0.982 & 1.153 & 5 \\ 

            \end{tabular}
            }
        \end{center}
        \caption{Wyniki dla przypadków testowych z pliku \textit{gapb.txt}. (*) $cap / cap_{max}$ przedstawia stosunek użytej pojemności maszyny 
            w wyznaczonym rozwiązaniu do maksymalnej pojemności maszyny. W tabeli przedstawiono minimalny, średni i maksymalny stosunek.}
        \label{wyniki-b}
    \end{table}

    \begin{table}[H]
        \begin{center}
            \resizebox{\textwidth}{!}{%
            \begin{tabular}{c|c||c|c|c|c|c||c|c|c|c|c|c}
                      &       & \multicolumn{5}{c||}{Wynik dokładny} & \multicolumn{6}{c}{Aproksymacja} \\ \cline{3-7} \cline{8-13}
                $|M|$ & $|J|$ & koszt & czas (s) & \multicolumn{3}{c||}{$cap / cap_{max}$ (*)} & koszt & czas (s) & \multicolumn{3}{c|}{$cap / cap_{max}$ (*)} & \# iteracji \\ \cline{5-7} \cline{10-12} 
                  &  &       &          & min & avg & max                         &       &          & min & avg & max                                        & \\
                \hline
                \hline

                5 & 100 & 1931.0 & 2.691 & 0.991 & 0.997 & 1.0 & 1904.0 & 0.0373 & 1.0 & 1.018 & 1.06 & 5 \\ 
                5 & 200 & 3456.0 & 181.1441 & 1.0 & 1.0 & 1.0 & 3441.0 & 0.0544 & 0.98 & 1.005 & 1.02 & 5 \\ 
                10 & 100 & --- & --- & --- & --- & --- & 1375.0 & 0.0799 & 0.839 & 1.02 & 1.158 & 5 \\ 
                10 & 200 & --- & --- & --- & --- & --- & 2772.0 & 0.1191 & 0.971 & 1.023 & 1.075 & 5 \\ 
                20 & 100 & --- & --- & --- & --- & --- & 1200.0 & 0.174 & 0.81 & 1.053 & 1.322 & 7 \\ 
                20 & 200 & --- & --- & --- & --- & --- & 2365.0 & 0.2369 & 0.87 & 1.02 & 1.126 & 5 \\ 

            \end{tabular}
            }
        \end{center}
        \caption{Wyniki dla przypadków testowych z pliku \textit{gapc.txt}. (*) $cap / cap_{max}$ przedstawia stosunek użytej pojemności maszyny 
            w wyznaczonym rozwiązaniu do maksymalnej pojemności maszyny. W tabeli przedstawiono minimalny, średni i maksymalny stosunek.}
        \label{wyniki-c}
    \end{table}

    \begin{table}[H]
        \begin{center}
            \resizebox{\textwidth}{!}{%
            \begin{tabular}{c|c||c|c|c|c|c||c|c|c|c|c|c}
                      &       & \multicolumn{5}{c||}{Wynik dokładny} & \multicolumn{6}{c}{Aproksymacja} \\ \cline{3-7} \cline{8-13}
                $|M|$ & $|J|$ & koszt & czas (s) & \multicolumn{3}{c||}{$cap / cap_{max}$ (*)} & koszt & czas (s) & \multicolumn{3}{c|}{$cap / cap_{max}$ (*)} & \# iteracji \\ \cline{5-7} \cline{10-12} 
                  &  &       &          & min & avg & max                         &       &          & min & avg & max                                        & \\
                \hline
                \hline

                5 & 100 & --- & --- & --- & --- & --- & 6230.0 & 0.0257 & 0.969 & 1.027 & 1.078 & 5 \\ 
                5 & 200 & --- & --- & --- & --- & --- & 12663.0 & 0.0359 & 0.99 & 1.009 & 1.029 & 3 \\ 
                10 & 100 & --- & --- & --- & --- & --- & 6170.0 & 0.0576 & 0.858 & 1.036 & 1.176 & 5 \\ 
                10 & 200 & --- & --- & --- & --- & --- & 12258.0 & 0.1693 & 0.954 & 1.019 & 1.077 & 7 \\ 
                20 & 100 & --- & --- & --- & --- & --- & 5890.0 & 0.1443 & 0.617 & 1.058 & 1.444 & 6 \\ 
                20 & 200 & --- & --- & --- & --- & --- & 11918.0 & 0.3548 & 0.888 & 1.036 & 1.204 & 6 \\ 
                
            \end{tabular}
            }
        \end{center}
        \caption{Wyniki dla przypadków testowych z pliku \textit{gapd.txt}. (*) $cap / cap_{max}$ przedstawia stosunek użytej pojemności maszyny 
            w wyznaczonym rozwiązaniu do maksymalnej pojemności maszyny. W tabeli przedstawiono minimalny, średni i maksymalny stosunek.}
        \label{wyniki-d}
    \end{table}

    \section{Analiza wyników}

    Wyniki przedstawione w poprzedniej sekcji pokazują, że każde rozwiązanie wyznaczone przy użyciu algorytmu aproksymacyjnego jest 
    nie gorsze niż rozwiązanie optymalne. Pojemność maszyn nie została przekroczona o więcej niż 2 razy. Dodatkowo możemy zauważyć 
    znaczącą różnicę w czasie, jakiego algorytmy potrzebowały na znalezienie rozwiązania. Czas jakiego potrzebował algorytm aproksymacyjny 
    nie przekroczył sekundy dla żadnego z testowanych przypadków, dla większości z nich był nawet 2 rzędy wielkości mniejszy. 
    Algorytm dokładny potrzebował dużo więcej czasu. Dla niektórych przypadków nie udało się wyznaczyć rozwiązania ze względu na zbyt długi czas 
    działania algorytmu. Co do przekroczenia pojemności maszyny, to widzimy też, że w żadnym z testowanych zadań największy stosunek pojemności 
    maszyny do pojemności maksymalnej nie zbliżył się nawet do $2$. Algorytm aproksymacyjny potrzebował około 2 - 7 iteracji na 
    znalezienie rozwiązania w zależności od zadania. 

\end{document}