\documentclass{article}
\title{Metody optymalizacji. Lista 2}
\author{Piotr Berezowski, 236749}

\usepackage{polski}
\usepackage[utf8]{inputenc}
\usepackage{enumerate}
\usepackage{graphicx}
\usepackage{subfig}
\usepackage{cases}
\usepackage{mathtools}
\usepackage{float}
\usepackage{cleveref}
\graphicspath{ {./src/} }

\begin{document}
	\maketitle
	\pagenumbering{gobble}
	\newpage
	\pagenumbering{arabic}
	
	\section{Zadanie 1}
	\subsection{Opis zadania}

    Opisany w zadaniu problem należy sformułować w postaci zadania programowania całkowitoliczbowego. Następnie korzystając pakietu JuMP 
    rozwiązać problem przy użyciu solvera GLPK lub Cbc.
    
    Chcemy zebrać dane liczbowe na temat $m$ różnych cech populacji. Dane zapisane są w chmurze w $n$ różnych miejscach. Czas potrzebny na przeszukanie 
    $i$-tego miejsca jest równy $T_i$. Zakładamy, że czas ten nie zależy od liczby odczytywanych cech. Dane dotyczące każdej z cech mogą znajdować się 
    w kilku miejscach. Chcemy wyznaczyć zbiór miejsc, które trzeba przeszukać, aby zebrać informacje o wszystkich cechach jednocześnie minimalizując 
    całkowity czas potrzebny na odczytanie tych cech.

    \subsection{Model}

    Zdefiniujmy zbiór $n$ serwerów jako $S = \{1, 2, \dots, m\}$, oraz zbiór $m$ cech jako $C = \{1, 2, \dots, n\}$. Niech 
    $T_j$, $j \in S$, oznacza czas potrzebny na przeszukanie $j$-tego serwera. Niech $q_{ij}, \ i \in C, \ j \in S$ przyjmuje wartość $1$ 
    jeśli dane dotyczące $i$-tej cechy znajdują się na $j$-tym serwerze. W przeciwnym przypadku $q_{ij} = 0$.

    \subsubsection{Zmienne decyzyjne}

        Zmienne decyzyjne określające które serwery należy przeszukać definiujemy jako:

        \begin{equation*}
            x_{j} =
            \begin{cases}
                1, & \text{jeśli serwer $j$ będzie przeszukany} \\
                0, & \text{w przeciwnym przypadku}
            \end{cases}
            \text{, gdzie $j \in S$}
        \end{equation*}

    \subsubsection{Ograniczenia}

    Ograniczenia modelują następujące sytuacje:
        \begin{itemize}
            \item Należy odczytać informacje dotyczące wszystkich cech:
                $$\sum_{j \in S} x_{j} q_{ij} \geq 1, \ i \in C$$

            \item Binarność zmeinnych decyzyjnych:
                $$x_j \in \{0, 1\}, j \in S$$

        \end{itemize}

    \subsubsection{Funkcja celu}

        Funkcja celu przyjmuje postać:

        $$\sum_{j \in S} x_j T_j$$
    
    \subsection{Rozwiązanie}
    
    Implementacja modelu wraz z przykładowymi danymi znajduje się w pliku \textit{zad1.jl}.

    Rozwiązano problem dla następujących danych:

    \begin{itemize}
        \item Ilość cech: $7$
        \item Ilość serwerów: $6$
        \item Czasy odczytu dla kolejnych serwerów: $1, 2, 5, 2, 4, 10$
        \item Wartości $q_{ij}$:
        \begin{table}[H]
            \begin{center}
                \begin{tabular}{c||c|c|c|c|c|c||c}
                     & \textbf{1} & \textbf{2} & \textbf{3} & \textbf{4} & \textbf{5} & \textbf{6} & \textbf{S} \\ 
                    \hline
                    \hline
                    \textbf{1} & 1 & 0 & 0 & 1 & 0 & 0 &  \\
                    \hline
                    \textbf{2} & 0 & 1 & 0 & 0 & 1 & 0 &  \\
                    \hline
                    \textbf{3} & 0 & 0 & 1 & 0 & 0 & 1 &  \\
                    \hline
                    \textbf{4} & 1 & 0 & 0 & 1 & 0 & 0 &  \\
                    \hline
                    \textbf{5} & 0 & 1 & 0 & 0 & 1 & 0 &  \\
                    \hline
                    \textbf{6} & 0 & 0 & 1 & 0 & 0 & 1 &  \\
                    \hline
                    \textbf{7} & 0 & 0 & 0 & 0 & 0 & 1 &  \\
                    \hline
                    \hline
                    \textbf{C} &  &  &  &  &  &  &  \\
                \end{tabular}
                \caption{Macierz $q$}
                \label{tab1}
            \end{center}
        \end{table}
    \end{itemize}

    Dla podanych danych minimalny czas przeszukiwania serwerów jest równy $13$. Należy przeszukać serwery: $1, 2$ i $6$.

    \section{Zadanie 2}
	\subsection{Opis zadania}

    Opisany w zadaniu problem należy sformułować w postaci zadania programowania całkowitoliczbowego. Następnie korzystając pakietu JuMP 
    rozwiązać problem przy użyciu solvera GLPK lub Cbc.

    Niech $P_{ij}$ będzie $j$-tym podprogramem obliczania funkcji $i$ należącym do biblioteki podprogramów 
    ($i \in \{1,\dots ,m\},\ j \in \{1,\dots,n\}$).
    Podprogram $P_{ij}$ zajmuje $r_{ij}$ komórek pamięci i potrzeba na jego wykonanie $t_{ij}$ jednostek czasu.
    Dla zadanego zbioru funkcji $I, \ I \subseteq \{1, \dots, m\}$ należy ułożyć program $P$ obliczający zbiór tych funkcji, w taki sposób, 
    żeby zajmował on nie więcej niż $M$ komórek pamięci, oraz jego czas wykonania był minimalny.

    \subsection{Model}
    
      
    \subsubsection{Zmienne decyzyjne}

    \begin{itemize}
        \item Zmienne określające czy podprogram $P_{ij}$ wchodzi w skład programu $P$:
        \begin{equation*}
            x_{ij} =
            \begin{cases}
                1, & \text{jeśli podprogram $P_{ij}$ należy do programu $P$} \\
                0, & \text{w przeciwnym przypadku}
            \end{cases}
            \text{, $i \in \{1,\dots, m\},\ j \in \{1,\dots, n\}$}
        \end{equation*}  
    \end{itemize}


    \subsubsection{Ograniczenia}

    Ograniczenia modelują następujące sytuacje:
    \begin{itemize}
        \item Wybieramy tylko jeden podprogram dla każdej funkcji obliczanej w $P$:
            $$\sum_{j \in \{1, \dots, n\}} x_{ij} = 1, \ i \in I$$
        
        \item Maksymalne zużycie pamięci nie może przekroczyć wartości $M$:
            $$\sum_{i \in \{1, \dots, m\}} \sum_{j \in \{1, \dots, n\}} x_{ij} r_{ij} \leq M$$

        \item Binarność zmiennych decyzyjnych:
            $$x_{ij} \in \{0, 1\},\ i \in \{1,\dots, m\},\ j \in \{1,\dots, n\}$$
    \end{itemize}
    
    \subsubsection{Funkcja celu}

    Funkcja celu przyjmuje postać:
        $$\sum_{i \in \{1,\dots, m\}} \sum_{j \in \{1,\dots, n\}} x_{ij} t_{ij}$$

    \subsection{Rozwiązanie}

    Implementacja modelu wraz z przykładowymi danymi znajduje się w pliku \textit{zad2.jl}.

    Rozwiązano problem dla następujących danych:

    \begin{itemize}
        \item Ilość funkcji w bibliotece: $m = 5$
        \item Ilość podprogramów obliczających poszczególne funkcje: $n = 4$
        \item Wartości $r_{ij}$:
            \begin{table}[H]
                \begin{center}
                    \begin{tabular}{c||c|c|c|c}
                        & \textbf{1} & \textbf{2} & \textbf{3} & \textbf{4} \\ 
                        \hline
                        \hline
                        \textbf{1} & 10 & 1 & 1 & 2 \\
                        \hline
                        \textbf{2} & 2 & 2 & 3 & 1 \\
                        \hline
                        \textbf{3} & 2 & 3 & 1 & 1 \\
                        \hline
                        \textbf{4} & 2 & 3 & 1 & 1 \\
                        \hline
                        \textbf{5} & 2 & 3 & 1 & 1 \\
                    \end{tabular}
                    \caption{Wartości $r_{ij},\ i \in \{1, \dots, 5\},\ j \in \{1, \dots, 4\}$}
                    \label{tab2}
                \end{center}
            \end{table}
        \item Wartości $t_{ij}$:
            \begin{table}[H]
                \begin{center}
                    \begin{tabular}{c||c|c|c|c}
                        & \textbf{1} & \textbf{2} & \textbf{3} & \textbf{4} \\ 
                        \hline
                        \hline
                        \textbf{1} & 1 & 4 & 5 & 2 \\
                        \hline
                        \textbf{2} & 1 & 2 & 2 & 5 \\
                        \hline
                        \textbf{3} & 4 & 2 & 5 & 6 \\
                        \hline
                        \textbf{4} & 4 & 2 & 5 & 6 \\
                        \hline
                        \textbf{5} & 4 & 2 & 5 & 6 \\
                    \end{tabular}
                    \caption{Wartości $t_{ij},\ i \in \{1, \dots, 5\},\ j \in \{1, \dots, 4\}$}
                    \label{tab2}
                \end{center}
            \end{table}
        \item Maksymalna dostępna ilość pamięci dla programu: $M = 10$
        \item Zbiór funkcji z których składa się program $P$: $I = \{1, 2, 3, 4, 5\}$
    \end{itemize}

    Dla podanych danych otrzymano rozwiązanie:

    \begin{table}[H]
        \begin{center}
            \begin{tabular}{c|c}
                \textbf{Funkcja} & \textbf{Podprogram} \\ 
                \hline
                1 & 4 \\
                2 & 1 \\
                3 & 2 \\
                4 & 1 \\
                5 & 3 \\
            \end{tabular}
            \caption{Rozwiązanie zadania. Całkowity czas programu jest równy $14$.}
            \label{tab3}
        \end{center}
    \end{table}

    \section{Zadanie 3}
	\subsection{Opis zadania}

    Opisany w zadaniu problem należy sformułować w postaci zadania programowania całkowitoliczbowego. Następnie korzystając z pakietu JuMP 
    rozwiązać problem przy użyciu solvera GLPK lub Cbc.

    Dany jest zbiór zadań $Z={1, \dots, n}$, które mają być wykonywane na trzech procesorach $P_1, P_2$ i $P_3$. Zakładamy, że:
    \begin{itemize}
        \item każdy procesor może wykonywać w danym momencie tylko jedno zadanie,
        \item każde zadanie musi być wykonywana najpierw na procesorze $P_1$ następnie na procesorze $P_2$ i na końcu na procesorze $P_3$,
        \item kolejność wykonywania zadań na wszystkich trzech procesorach jest taka sama.
    \end{itemize}
    Dla każdego zadania $i \in Z$ są zadane czasy trwania $a_i, b_i$ oraz $c_i$ odpowiednio na procesorach $P_1$, $P_2$ i $P_3$. 
    Wszystkie dane są dodatnimi liczbami całkowitymi. Każdy harmonogram jest jednoznacznie określony przez pewną permutację
    $\pi = (\pi(1), \dots, \pi(n))$ zadań należących do zbioru $Z$. Niech $C_{\pi(k)}$ oznacza czas zakończenia $k$-go zadania na 
    procesorze $P_3$ dla permutacji $\pi$. Celem jest wyznaczenie permutacji $\pi$ takiej, że:
    $$C_{\max} = C_{\pi(n)} \rightarrow \min$$.
       
    \subsection{Model}

    Niech $P = \{P_1, P_2, P_3\}$ będzie zbiorem procesorów. Niech $t_{pj}$ określa czas trwania zadania $j$ na procesorze $p$, wtedy 
    $t_{P_1j} = a_j$, $t_{P_2j} = b_j$ oraz $t_{P_3j} = c_j$. Niech $B$ będzie bardzo dużą liczbą.

    \subsubsection{Zmienne decyzyjne}

    Wprowadźmy następujące zmienne:
    \begin{itemize}
        \item Zmienne określające moment rozpoczęcia $j$-tego zadania na $i$-tym procesorze:
            $$x_{ij} \geq 0, \ i \in P, \ j \in Z$$
        \item Zmienne pomocnicze określające czy zadanie $i$ jest wykonywane przed zadaniem $j$:
            \begin{equation*}
                y_{ij} =
                \begin{cases}
                    1, & \text{jeśli zadanie $i$ jest wykonywane przed zadaniem $j$} \\
                    0, & \text{w przeciwnym przypadku}
                \end{cases}
                \text{, $i \in Z,\ j \in Z,\ i \neq j$}
            \end{equation*}  
        \item Zmienna pomocnicza modelująca moment zakończenia ostatniego zadania na procesorze $P_3$:
            $$M \geq 0$$
    \end{itemize}
        
    \subsubsection{Ograniczenia}

        Ograniczenia modelują następujące sytuacje:
        \begin{itemize}
            \item Zachowanie kolejności procesorów - zadanie musi zostać wykonane na poprzednim procesorze, aby mogło być liczone na kolejnym:
        \end{itemize}
                $$x_{P_1j} + t_{P_1j} \leq x_{P_2j},\ j \in Z,$$
                $$x_{P_2j} + t_{P_2j} \leq x_{P_3j},\ j \in Z$$
        
        \begin{itemize}
            \item Zadania nie nakładają się - każdy procesor może obliczać jedno zadanie w tym samym czasie:
        \end{itemize}
                $$x_{pi} - x_{pj} + B y_{ij} \geq t_{pj},\ p \in P,\ i \in Z,\ j \in Z$$
                $$x_{pj} - x_{pi} + B (1 - y_{ij}) \geq t_{pi},\ p \in P,\ i \in Z,\ j \in Z$$
        
        \begin{itemize}
            \item Ograniczenia modelujące wartość zmiennej $M = \max_{j \in Z} x_{P_3j} + t_{P_3j}$:
        \end{itemize}
                $$x_{P_3j} + d_{P_3j} \leq M,\ j \in Z$$

        \begin{itemize}
            \item Binarność zmiennych pomocniczych $y_{ij}$:
        \end{itemize}
                $$y_{ij} \in \{0, 1\},\ i \in Z,\ j \in Z,\ i \neq j$$

    \subsubsection{Funkcja celu}

        Funkcja celu przyjmuje postać: 
        $$M$$

    \subsection{Rozwiązanie}
    
    Implementacja modelu wraz z przykładowymi danymi znajduje się w pliku \textit{zad3.jl}.

    Rozwiązano problem dla następujących danych:

    \begin{itemize}
        \item Ilość zadań: $n = 5$
        \item Czasy wykonania zadań na poszczególnych procesorach:
            \begin{table}[H]
                \begin{center}
                    \begin{tabular}{c||c|c|c|c|c||c}
                         & \textbf{1} & \textbf{2} & \textbf{3} & \textbf{4} & \textbf{5} & \textbf{Z} \\ 
                        \hline
                        \hline
                        \textbf{$P_1$} & 1 & 2 & 3 & 4 & 5 &  \\
                        \hline
                        \textbf{$P_2$} & 2 & 3 & 4 & 3 & 2 &  \\
                        \hline
                        \textbf{$P_3$} & 1 & 2 & 3 & 1 & 2 &  \\
                        \hline
                        \hline
                        \textbf{$P$} &  &  &  &  &  &  \\
                    \end{tabular}
                    \caption{Wartości $t_{ij},\ i \in P,\ j \in Z$}
                    \label{tab4}
                \end{center}
            \end{table}
    \end{itemize}

    Dla podanych danych otrzymano rozwiązanie:
    $\pi = (2, 4, 3, 1, 5)$

\end{document}