\documentclass{article}
\title{Metody optymalizacji. Lista 1}
\author{Piotr Berezowski, 236749}

\usepackage{polski}
\usepackage[utf8]{inputenc}
\usepackage{enumerate}
\usepackage{graphicx}
\usepackage{subfig}
\usepackage{cleveref}
\usepackage{cases}
\usepackage{mathtools}
\usepackage{float}
\graphicspath{ {./src/} }

\begin{document}
	\maketitle
	\pagenumbering{gobble}
	\newpage
	\pagenumbering{arabic}
	
	\section{Zadanie 1}
	\subsection{Opis zadania}

    Zadanie polegało na zapisaniu modelu w \textit{GNU MathProg} i następnie użyciu \textit{glpsol} do rozwiązania go i sprawdzenia dla jakich $n$ zwracany wynik 
    jest dokładny do co najmniej 2 cyfr. Model miał opisywać jeden z testów używanych do weryfikowania dokładności i odporności algorytmów LP.
    
    \subsection{Model}

    Szukamy:
    $$\min{c^T x}$$
    Przy spełnionych warunkach:
    $$Ax = b,\ x \geq 0$$
    gdzie
    $$a_{ij} = \frac{1}{i + j - 1},\ i, j = 1, \dots, n$$
    $$c_i = b_i = \sum_{j=1}^{n} \frac{1}{i + j - 1},\ i = 1, \dots, n$$
    
    Wiemy, że rozwiązaniem opisanego wyżej zagadnienia jest $x_i = 1$, $i = 1, \dots, n$, a macierz $A$ występująca w zadaniu powoduje złe uwarunkowanie 
    zadania już dla małych n.
    
    \subsection{Rozwiązanie}

    Zaimplementowany model znajduje się w dołączonych do sprawozdania kodach źródłowych w pliku \textit{zad1.mod}. Został on podzielony na dwie części, gdzie jedna 
    zawiera opis modelu, a druga dane - wartość $n$ dla której chcemy znaleźć rozwiązanie. 

    Poniższa tabela \ref{tab1} zawiera wartość błędu względnego obliczonego dla kolejnych wartości $n$. Błąd względny był obliczany według wzoru:
    $$\delta = \frac{||x - \tilde{x}||_2}{||x||_2}$$

    \begin{table}[H]
		\begin{center}
			\begin{tabular}{c||c}
				\textbf{n} & \textbf{Błąd względny ($\delta$)} \\
				\hline
                1 & $0$ \\
                2 & $1.053 * 10^{-15}$ \\
                3 & $3.67158 * 10^{-15}$ \\
                4 & $3.27016 * 10^{-13}$ \\
                5 & $3.3514 * 10^{-12}$ \\
                6 & $6.83336 * 10^{-11}$ \\
                7 & $1.67869 * 10^{-8}$ \\
                8 & $0.514059$ \\
                9 & $0.682911$ \\
                10 & $0.990388$ \\
			\end{tabular}
            \caption{Wyniki dla zadania 1}
            \label{tab1}
		\end{center}
	\end{table}	

    Widzimy, że wyniki są obliczane z akceptowalną dokładnością dla wartości $n \leq 7$. Powyżej tej wartości $n$ błąd względny jest większy niż $0.01$.


	\section{Zadanie 2}
	\subsection{Opis zadania}

    Celem zadania było zapisanie modelu programowania liniowego w \textit{GNU MathProg} dla przedstawionego problemu i rozwiązanie go używając \textit{glpsol}. 
    
    \subsection{Model}

        Zdefiniujmy zbiór miast, w których zlokalizowane są przedstawicielstwa firmy $M = \{$Warszawa, Gdańsk, $\dots$, Budapeszt$\}$ oraz zbiór dostępnych typów camperów 
        $T = \{$Standard, VIP$\}$. Niech $\delta_{ij}$ oznacza odległość między i-tą i j-tą lokalizacją firmy.

        Następnie niech $d_t(m)$ oznacza niedobór campera typu $t$ w mieście $m$, a $s_t(m)$ oznacza nadmiar campera typu $t$ w mieście $m$.

        Na końcu zdefiniujmy $c_t$ jako koszt przemieszczenia kampera typu $t$ o 1 km.

    \subsubsection{Zmienne decyzyjne}

        Zmienne decyzyjne wyznaczające ilość samochodów typu $t$ transportowanych z lokacji $i$ do $j$ definiujemy jako:

        $$x_{t_{ij}} \geq 0, \ t \in T, \ i,j \in M$$

    \subsubsection{Ograniczenia}

        \begin{itemize}
            \item Ilość camperów typu $t$, która wyjeżdża z miasta $m$ nie może przekroczyć podaży (nadmiaru) camperów w tym mieście:
                $$\sum_{j \in M} x_{t_{mj}} \leq s_t(m), \ t \in T, m \in M$$

            \item Ilość wszystkich camperów, które są dostarczane do miasta $m$ musi być co najmniej równa niedoborowi wszystkich camperów w tym mieście:
                $$\sum_{i \in M}\sum_{t \in T} x_{t_{im}} \geq \sum_{t \in T}d_t(m), \ m \in M$$

            \item Ilość camperów typu VIP, które są dostarczane do miasta $m$ musi być co najmniej równa niedoborowni camperów typu VIP w tym mieście:
                $$\sum_{i \in M} x_{t_{im}} \geq d_{t}(m), \ t = VIP, m \in M$$
        \end{itemize}

    \subsubsection{Funkcja celu}

        Funkcja celu opisuje taki plan transportu camperów, którego koszt całkowity jest najmniejszy. Przyjmuje następującą postać:

        $$\min \sum_{t \in T} \sum_{i \in M} \sum_{j \in M} x_{t_{ij}} c_t \delta_{ij}$$

        Wyrażenie $c_t \delta_{ij}$ określa koszt transportu jednego campera typu $t$ z miasta $i$ do miasta $j$.
    
    \subsection{Rozwiązanie}
    
        Implementacja modelu znajduje się w dołączonym pliku \textit{zad2.mod}. Dane dla przedstawionego zadania znajdują się w pliku \textit{zad2.dat}.
        
        Tabela \ref{tab2} zawiera optymalny plan transportu dla zdefiniowanych danych. Koszt całkowity dla wyznaczonego planu wyniósł $20686.2$.
        
        \begin{table}[H]
            \begin{center}
                \begin{tabular}{c|c|c|c}
                    \textbf{Źródło} & \textbf{Cel} & \textbf{\#VIP} & \textbf{\#Standard} \\
                    \hline
                    Warszawa  &  Gdańsk  & 14  & 0  \\ 
                    Gdańsk  &  Gdańsk  & 0  & 2  \\ 
                    Szczecin  &  Gdańsk  & 4  & 0  \\ 
                    Szczecin  &  Berlin  & 8  & 4  \\ 
                    Wrocław  &  Warszawa  & 0  & 4  \\ 
                    Wrocław  &  Wrocław  & 0  & 2  \\ 
                    Wrocław  &  Kraków  & 0  & 4  \\ 
                    Kraków  &  Wrocław  & 6  & 0  \\ 
                    Kraków  &  Koszyce  & 4  & 0  \\ 
                    Rostok  &  Berlin  & 0  & 2  \\ 
                    Rostok  &  Rostok  & 0  & 2  \\ 
                    Lipsk  &  Berlin  & 0  & 3  \\ 
                    Lipsk  &  Lipsk  & 0  & 3  \\ 
                    Lipsk  &  Praga  & 0  & 4  \\ 
                    Praga  &  Berlin  & 3  & 0  \\ 
                    Praga  &  Brno  & 7  & 0  \\ 
                    Brno  &  Brno  & 0  & 2  \\ 
                    Bratysława  &  Bratysława  & 0  & 4  \\ 
                    Bratysława  &  Budapeszt  & 0  & 4  \\ 
                    Koszyce  &  Kraków  & 0  & 4  \\ 
                    Budapeszt  &  Budapeszt  & 0  & 4  \\ 
                \end{tabular}
                \caption{Optymalny plan transportu. Koszt całkowity dla wyznaczonego planu transportu jest równy $20686.2$}
                \label{tab2}
            \end{center}
        \end{table}	    
    
        Wprowadzenie dodatkowego założenia co do całkowitoliczbowości zmiennych decyzyjnych $x_{t_{ij}}$ nie jest potrzebne.
    

    \section{Zadanie 3}
	\subsection{Opis zadania}

        Celem zadania było zapisanie modelu programowania liniowego w \textit{GNU MathProg} dla przedstawionego problemu i rozwiązanie go używając \textit{glpsol}. 
        Problem dotyczył planowania produkcji czterech mieszanek (produktów) w pewnej firmie. 
        Produkty są wytwarzane przy użyciu trzech rodzajów surowców. Służą one do produkcji dwóch produktów podstawowych. 
        Podczas wytwarzania produktów podstawowych powstają odpady, które mogą być dalej użyte do produkcji dwóch produktów drugorzędnych, 
        które wymagają również odpowiedniej ilości surowców. Odpady, które nie zostaną zużyte podczas produkcji produktów drugorzędnych muszą zostać 
        zniszczone.
        Należało wyznaczyć taki plan, który zmaksymalizuje zyski firmy.
    
    \subsection{Model}

        Zdefiniujmy zbiór mieszanek produkowanych przez firmę $M = \{A, B, C, D\}$, oraz zbiór surowców $S = \{s_1, s_2, s_3\}$.
        Niech $L_i, \ i \in S$ oznacza minimalną ilość $i$-tego surowca jaką firma musi kupić, a $U_i, \ i \in S$ oznacza maksymalną 
        ilość $i$-tego surowca jaką firma jest w stanie przetworzyć.

        Niech $w_{ij}, \ i \in S, \ j \in M$ oznacza współczynnik strat $i$-tego surowca dla $j$-tej mieszanki (współczynnik opisujący ilość powstałych odpadów).

        Niech $v_i, \ i \in M$ oznacza cenę 1 kg $i$-tej mieszanki.
        
        Niech $c_i, \ i \in S$ oznacza koszt za 1 kg $i$-tego surowca.

        Niech $u_{ij}, \ i \in S, \ j \in M$ oznacza koszt utylizacji odpadów powstałych z $i$-tego surowca przy produkcji $j$-tej mieszanki.

    \subsubsection{Zmienne decyzyjne}

        \begin{itemize}
            \item Zmienne decyzyjne wyznaczające ilość $i$-tego surowca, która została przeznaczona na produkcję $j$-tej mieszanki:
                $$x_{ij} \geq 0, \ i \in S, \ j \in M$$

            \item Zmienne decyzyjne wyznaczające ilość odpadów $i$-tego surowca, powstałych przy produkcji mieszanki $A$, przeznaczonych do 
                produkcji surowca $C$:
                $$y_{Ai} \geq 0, \ i \in S$$

            \item Zmienne decyzyjne wyznaczające ilość odpadów $i$-tego surowca, powstałych przy produkcji mieszanki $B$, przeznaczonych do 
                produkcji surowca $D$:
                $$y_{Bi} \geq 0, \ i \in S$$
        \end{itemize}


    \subsubsection{Ograniczenia}

        \begin{itemize}
            \item Górne i dolne ograniczenie na ilość surowca jaką firma musi zakupić:
                $$L_i \leq \sum_{j \in M} x_{ij} \leq U_i, \ i \in S$$

            \item Ograniczenia modelujące zmienne $y_{Ai}$ oraz $y_{Bi}$, które warunkują ilość odpadów przeznaczonych do dalszej produkcji:
            
                $$y_{Ai} \leq w_{iA} x_{iA}, \ i \in S$$
                $$y_{Bi} \leq w_{iB} x_{iB}, \ i \in S$$

            \item Ograniczenia warunkowane przez skład każdej z mieszanek:
                \begin{enumerate}
                    \item Mieszanka $A$:
                        $$x_{s_{1}A} \geq 0.2 \sum_{i \in S} x_{iA}$$
                        $$x_{s_{2}A} \geq 0.4 \sum_{i \in S} x_{iA}$$
                        $$x_{s_{3}A} \leq 0.1 \sum_{i \in S} x_{iA}$$
                    \item Mieszanka $B$:
                        $$x_{s_{1}B} \geq 0.1 \sum_{i \in S} x_{iB}$$
                        $$x_{s_{3}B} \leq 0.3 \sum_{i \in S} x_{iB}$$
                    \item Mieszanka $C$:
                        $$x_{s_{1}C} = 0.2 (x_{s_{1}C} + \sum_{i \in S} y_{Ai})$$
                        $$x_{s_{2}C} = 0$$
                        $$x_{s_{3}C} = 0$$
                    \item Mieszanka $D$:
                        $$x_{s_{1}D} = 0$$
                        $$x_{s_{2}D} = 0.3 (x_{s_{2}D} + \sum_{i \in S} y_{Bi})$$
                        $$x_{s_{3}D} = 0$$
                \end{enumerate}
        \end{itemize}

    \subsubsection{Funkcja celu}

        Funkcja zysku firmy będzie składać się z trzech składowych - kosztu zakupu surowców, kosztu utylizacji odpadów produkcyjnych oraz zysku ze 
        sprzedaży wytworzonych mieszanek. Całkowity zysk możemy więc zapisać jako:

        $$F = P - C_s - C_u$$ gdzie $P$ jest całkowitym zyskiem ze sprzedaży mieszanek, $C_s$ jest kosztem zakupu wszystkich surowców, a 
        $C_u$ jest kosztem utylizacji wszystkich odpadów powstałych podczas produkcji.

        Koszt zakupu wszystkich surowców zapisujemy jako:

        $$C_s = \sum_{i \in S} \sum_{j \in M} c_i x_{ij}$$

        Koszt utylizacji odpadów zapisujemy jako:

        $$C_u = \sum_{i \in S} \sum_{j \in \{A, B\}} u_{ij} (w_{ij} x_{ij} - y_{ji})$$

        Zysk ze sprzedaży mieszanek zapisujemy jako:

        $$P = P_A + P_B + P_C + P_D$$

        gdzie $P_i$ oznacza zysk uzyskany ze sprzedaży $i$-tej mieszanki.

        $$P_A = v_A \sum_{i \in S} (1-w_{iA}) x_{iA}$$
        $$P_B = v_B \sum_{i \in S} (1-w_{iB}) x_{iB}$$
        $$P_C = v_C (x_{s_{1}C} + \sum_{i \in S} y_{Ai})$$
        $$P_D = v_D (x_{s_{2}D} + \sum_{i \in S} y_{Bi})$$

        Naszym celem jest maksymalizacja zysku, a więc nasza funkcja celu będzie miała postać:
        $$\max{F} = \max{P - C_s - C_u}$$

    \subsection{Rozwiązanie}

        Implementacja modelu, razem z danymi znajduje się w dołączonym pliku \textit{zad3.mod}.
            
        W tabeli \ref{tab3} przedstawiono ilości poszczególnych surowców jakie powinny zostać użyte do produkcji mieszanek dla optymalnego planu produkcji.

        \begin{table}[H]
            \begin{center}
                \begin{tabular}{c||c|c|c|c|c}
                    \textbf{Surowiec} & \textbf{A} & \textbf{B} & \textbf{C} & \textbf{D} & \textbf{Razem} \\
                    \hline
                    1   & 1175.09   & 4725.03   & 99.8825   & 0   & 6000 \\ 
                    2   & 940.071   & 4059.93   & 0   & 0   & 5000 \\ 
                    3   & 235.018   & 3764.98   & 0   & 0   & 4000 \\
                \end{tabular}
                \caption{Ilość zakupionych surowców i ich rozkład między poszczególne produkty.}
                \label{tab3}
            \end{center}
        \end{table}	        

        W tabeli \ref{tab4} przedstawiono ilości odpadów jakie powstały w czasie produkcji mieszanek $A$ i $B$. Wartość w kolumnie \textit{Użyte} określa 
        ilość odpadów, jaka musi być użyta do produkcji produktów drugorzędnych, z kolei wartość w kolumnie \textit{Zniszczone} przedstawia ilość odpadów 
        jaką należy poddać utylizacji. 

        \begin{table}[H]
            \begin{center}
                \begin{tabular}{c|c||c|c|c}
                    \textbf{Produkt} & \textbf{Surowiec} & \textbf{Użyte} & \textbf{Zniszczone} & \textbf{Razem} \\ 
                    \hline
                    A   & 1   & 117.509   & 0   & 117.509 \\ 
                    A   & 2   & 188.014   & 0   & 188.014 \\ 
                    A   & 3   & 94.0071   & 0   & 94.0071 \\ 
                    A   & 1 + 2 + 3   & 399.53   & 0   & 399.53 \\ 
                    \hline
                    B   & 1   & 0   & 945.006   & 945.006 \\ 
                    B   & 2   & 0   & 811.986   & 811.986 \\ 
                    B   & 3   & 0   & 1882.49   & 1882.49 \\ 
                    B   & 1 + 2 + 3   & 0   & 3639.48   & 3639.48 \\
                \end{tabular}
                \caption{Ilość odpadów otrzymanych z produkcji mieszanek A i B.}
                \label{tab4}
            \end{center}
        \end{table}

        Optymalny zysk dla optymalnego rozwiązania jest równy $2986.89\$$.

    \section{Zadanie 4}
	\subsection{Opis zadania}

        Celem zadania było zapisanie modelu programowania całkowitoliczbowego w \textit{GNU MathProg} dla przedstawionego problemu i rozwiązanie go 
        używając \textit{glpsol}.

        Problem dotyczył układania planu zajęć dla studenta. Należało tak ułożyć plan, aby jak najlepiej wpasował się w preferencje studenta.
    
    \subsection{Model}

        Zdefiniujmy zbiór wymaganych kursów jako $C = \{$Algebra, Analiza, Fizyka, Chemia min., Chemia org.$\}$.
        
        Zdefiniujmy zbiór dostępnych grup dla każdego z kursów jako $G = \{$I, II, III, IV$\}$.
        
        Zdefiniujmy zbiór dni tygodnia jako $D = \{$Pn, Wt, Śr, Cz, Pt$\}$.
        
        Zdefiniujmy zbiór dostępnych grup treningowych jako $T = \{t_1, t_2, t_3\}$.

        Niech $W_{cg}$ będzie ilością punktów preferencyjnych przyznanych przez studenta 
        $g$-tej grupie $c$-tego kursu.

        Niech $p$ określa jakiś okres czasu. Wybierzmy wartość $p$ w taki sposób, 
        aby dobę dało się podzielić na całkowitą liczbę okresów długości $p$ oraz, aby 
        każdy z kursów również dał się podzielić na całkowitą ilość okresów długości $p$.

        Niech $n$ określa ilość okresów czasu długości $p$, które mieszczą się w dobie.

        Niech $Z_{cgdi}, \ c \in C, \ g \in G, \ d \in D \ i \in \{0,1,\dots,n\}$ będzie wartością binarną, która określa czy 
        $g$-ta grupa $c$-tego kursu odbywa się $d$-tego dnia podczas $i$-tego przedziału czasu.

        Analogicznie dla treningów, niech $S_{tdi}, \ t \in T, \ d \in D, \ i \in \{0,1,\dots,n\}$ będzie 
        wartością binarną, która określa czy trening $t$ odbywa się $d$-tego dnia podczas 
        $i$-tego przedziału czasu.

    \subsubsection{Zmienne decyzyjne}

        Zmienne decyzyjne opisujące plan zajęć studenta:
        \begin{equation*}
            x_{cg} =
            \begin{cases}
                1, & \text{jeśli grupa $g$ kursu $c$ zalicza się do planu studenta} \\
                0, & \text{w przeciwnym przypadku}
            \end{cases}
            \text{, gdzie $g \in G$, $c \in C$}
        \end{equation*}
        

        \begin{equation*}
            y_{t} =
            \begin{cases}
                1, & \text{jeśli trening $t$ zalicza się do plan studenta} \\
                0, & \text{w przeciwnym przypadku}
            \end{cases}
            \text{, gdzie $t \in T$}
        \end{equation*}
        
    \subsubsection{Ograniczenia}

        \begin{itemize}
            \item Wybieramy dokładnie jedną grupę z każdego kursu:
                $$\sum_{g \in G} x_{cg} = 1, \ c \in C$$
            
            \item Każdego dnia łączny czas zajęć studenta jest nie większy niż $l * p$:
                $$\sum_{c \in C} \sum_{g \in G} \sum_{i \in \{0,1,\dots,n\}} x_{cg} Z_{cgdi} \leq lp, \ d \in D$$
                
            \item Student ma $v$ przedziałów czasu wolnego między przedziałami $l$ oraz $u$ każdego dnia: 
                $$\sum_{c \in C} \sum_{g \in G} \sum_{i \in \{l, \dots, u-1\}} x_{cg} Z_{cgdi} \leq u - l - v, \ d \in D$$

            \item Zajęcia w planie nie nakładają się czasowo:
                $$\sum_{c \in C} \sum_{g \in G} x_{cg} Z_{cgdi} + \sum_{t \in T} y_{t} S_{tdi} \leq 1, \ d \in D, \ i \in \{0, \dots, n\}$$

            \item Student ma minimum jeden trening w tygodniu:
                $$\sum_{t \in T} y_{t} \geq 1$$

        \end{itemize}

    \subsubsection{Funkcja celu}

        Dla przedstawionego problemu, chcemy wyznaczyć taki plan zajęć, którego całkowita 
        ilość punktów preferencyjnych będzie maksymalna, przy spełnionych ograniczeniach 
        przedstawionych w poprzedniej sekcji.

        Możemy więc zapisać funkcję celu jako:

        $$\max{\sum_{c \in C} \sum_{g \in G} x_{cg} W_{cg}}$$

    \subsection{Rozwiązanie}
    
        Implementacja modelu, razem z danymi znajduje się w dołączonym pliku \textit{zad4.mod}.

        W tabeli \ref{tab5} przedstawiono optymalny plan zajęć spełniający ograniczenia 
        nałożone przez zadanie. Ilość punktów preferencyjnych dla tego planu jest 
        równa 37.

        \begin{table}[H]
            \begin{center}
                \begin{tabular}{c|c}
                    \textbf{Przedmiot} & \textbf{Wybrana grupa} \\ 
                    \hline
                    Algebra     & III \\
                    Analiza     & II \\
                    Fizyka      & IV \\
                    Chemia min. & I \\
                    Chemia org. & II 
                \end{tabular}
                \caption{Optymalny plan zajęć dla podstawowych ograniczeń.}
                \label{tab5}
            \end{center}
        \end{table}

        Dodatkowo, należało odpowiedzieć na pytanie, czy istnieje taki plan zajęć, 
        dla którego zajęcia zgrupowane byłyby w trzech dniach (poniedziałek, wtorek i czwartek) 
        oraz wszystkie miałyby ocenę wyższą niż 5.

        Nie udało się znaleźć plan spełniający dodatkowe ograniczenia.


\end{document}